% This file is a solution template for:

% - Giving a talk on some subject.
% - The talk is between 15min and 45min long.

% This is adapted from the example by Till Tantau <tantau@users.sourceforge.net>
% included as part of the beamer package in LaTeX
%
% In principle, this file can be redistributed and/or modified under
% the terms of the GNU Public License, version 2.
%
% However, this file is supposed to be a template to be modified
% for your own needs. For this reason, if you use this file as a
% template and not specifically distribute it as part of a another
% package/program, the author grants the extra permission to freely
% copy and  modify this file as you see fit and even to delete this
% copyright notice. 

\usemodule
  [complexslides]
  [
    %%%%%%%%%%%%%%%%%%%%%%%%%%%%%%%%%%%%
    %% Various configuration options. %%
    %% If an option is not specified, %%
    %% `default` value is used.       %%
    %%%%%%%%%%%%%%%%%%%%%%%%%%%%%%%%%%%%
    % fontscheme=default,
    % colorscheme=default, % cambria
    layout=right-toc, 
    % pagestyle=fuzzy,
    % setup=default,
    % itemize=default,
    % floats=default,
    marking=countdown, % progressbar
    % titlepage=default,
    extras={framedtext, % provides \startalert ... \stopaltert
                        % and \startresult ... \stopresult environment
           },
  ]

% In order to use sample images distributed with ConTeXt
\setupexternalfigures[location={local,global,default}] 

\starttext

\setvariables
  [talk]
  [
    title={An introduction to making slides using complexslides},
    shorttitle={Making slides using complexslides},
    author={First Author and Second Author},
    shortauthor={F.~Author & S.~Author},
    institute={ConTeXt University},
    thanks={Acknowledgements: ConTeXt mailing list},
    location={Someplace on Earth (Jan 29, 2012)},
  ]

% Since this a solution template for a generic talk, very little can
% be said about how it should be structured. However, the talk length
% of between 15min and 45min and the theme suggest that you stick to
% the following rules:  

% - Exactly two or three sections (other than the summary).
% - At *most* three subsections per section.
% - Talk about 30s to 2min per slide. So there should be between about
%   15 and 30 slides, all told.


\startslide[title={Make Titles Informative}]
  % - A title should summarize the slide in an understandable fashion
  %   for anyone how does not follow everything on the slide itself.

  \startitemize
    \item Use bullets points when appropriate.
    \item Use pictures when possible
    \item Do not put too much information on one slide
  \stopitemize
\stopslide

\startslide[title={A Dutch Cow}]
  % To include a full slide picute.
  \externalfigure[cow]
\stopslide

\startslidemakeup
  An important point
\stopslidemakeup

\startslide[title={The main result}]
  \startalert
    Problem Statement and why it is difficult. Some more text to just fill up
    space to show how the environment looks with a multi-line text.
  \stopalert

  \startresult
    Main result and its cool features. Some more text to just fill up
    space to show how the environment looks with a multi-line text.
  \stopresult
\stopslide

\startslide[title={Summary}]

  \startitemize
    \item The {\em first main message} of your talk in one or two lines.
    \item The {\em second main message} of your talk in one or two lines.
    \item Perhaps a {\em third message}, but not more than that.
  \stopitemize
\stopslide

\startslidemakeup
  Thank you
\stopslidemakeup

\stoptext

