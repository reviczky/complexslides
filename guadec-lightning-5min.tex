% This file is a solution template for:

% - Introducing another speaker.
% - Talk length is about 2min.
% - Style is informal.

% This is adapted from the example by Till Tantau <tantau@users.sourceforge.net>
% included as part of the beamer package in LaTeX
%
% In principle, this file can be redistributed and/or modified under
% the terms of the GNU Public License, version 2.
%
% However, this file is supposed to be a template to be modified
% for your own needs. For this reason, if you use this file as a
% template and not specifically distribute it as part of a another
% package/program, the author grants the extra permission to freely
% copy and  modify this file as you see fit and even to delete this
% copyright notice. 

\usemodule[complexslides][marking=progressbar,colorscheme=gnome,fontscheme=gnome,itemize=gnome,pagestyle=fuzzy]

% For a short talk, you can use the \type{markdown} module
% to create a presentation.
%
% \usemodule[markdown]
%
% The markdown module uses chapter as the top level header. Change that to
% section.
% 
% \definestructurelevels
%   [markdown]
%   [
%     section,
%     subsection,
%     subsubsection,
%     subsubsubsection,
%     subsubsubsubsection,
%   ]

% The markdown module does not handle all markdown features correctly.
% If you need proper support of markdown, use \type{pandoc} to convert
% markdown to TeX.

\usemodule[filter]

\defineexternalfilter
  [markdown]
  [filter={pandoc -f markdown -t context --number-sections -o \externalfilteroutputfile}]

% In order to use sample images distributed with ConTeXt
\setupexternalfigures[location={local,global,default}]


\starttext

% FIXME: startuseMPgraphic
\setvariables
  [talk]
  [
    title={Creating a {\switchtobodyfont[16pt]\GNOME{GNOMEVerticalTM}} logo Font \blank[2*line] Adam Reviczky},
    shorttitle={GNOME Logo Font},
    location={\midaligned{\hbox{\scale[lines=4]{\startMPcode input guadec-logo.mp; \stopMPcode}}}},
  ]

\startmarkdown
# Creating a "GNOME logo" Font

## Experience and achievements

\switchtobodyfont[16pt]

\blank[3*line]

- GNOME Brand Guideline (svg/eps)
- editing with inkscape/fontforge
- installable otf font
- pstoedit for Metapost

\switchtobodyfont[9pt]

# Fontforge magic

## Concerning today's talk

\startplacefigure%[location=right]
  \framed[frame=on,rulethickness=2pt]{\scale[factor=225]{\externalfigure[fontforge]}}
\stopplacefigure

\startplacefigure%[location=right]
  \centerbox{\framed[frame=on,rulethickness=2pt]{\scale[factor=150]{\externalfigure[charmap]}}}
\stopplacefigure

# Enumerations within \TeX

## Current affiliation of Speaker

\startplacefigure[location=right]
  \framed[frame=on,rulethickness=2pt]{\scale[factor=120]{\externalfigure[gnomefoot]}}
\stopplacefigure

\typefile{gnomefoot.tex}

# Enumerations within LibreOffice

## Current affiliation of Speaker

\startplacefigure%[location=right]
  \framed[frame=on,rulethickness=2pt]{\scale[factor=225]{\externalfigure[LO]}}
\stopplacefigure

# Metapost Logo

## Concerning today's talk

\switchtobodyfont[16pt]

- [http://www.tlhiv.org/mppreview/](http://www.tlhiv.org/mppreview/)

\startplacefigure%[location=right]
  \framed[frame=on,rulethickness=2pt]{\scale[factor=100]{\externalfigure[mp]}}
\stopplacefigure

\switchtobodyfont[9pt]

# Information

## Concerning today's talk

\switchtobodyfont[16pt]

\blank[2*line]

- [http://gist.github.com/reviczky](http://gist.github.com/reviczky)
- [revadam@gnome.org](mailto:revadam@gnome.org)

\blank[line]

- suggestions/improvements welcome

\blank[2*line]

\midaligned{\tfb Thanks!}

\switchtobodyfont[9pt]

\stopmarkdown

\stoptext
